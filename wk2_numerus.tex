\documentclass{beamer}
\usetheme{metropolis}           % Use metropolis theme
\title{Numerus: Theorie}
\date{\today}
\author{Peter W. Smith}
% \institute{Centre for Modern Beamer Themes}

\usepackage{polyglossia}
\setmainlanguage{german}

\usepackage[
   backend=biber,
   style=authoryear,
   natbib,
   url=false,
   doi=true,
   eprint=false
]{biblatex}
% \addbibresource{/home/pwsmith/Dropbox/Ducks/biblio.bib}
\addbibresource{~/Dropbox/Ducks/biblio.bib}


\resetcounteronoverlays{exx}
\setbeamertemplate{caption}[numbered]


\definecolor{dgreen}{HTML}{028912}
%\setbeamertemplate{caption}[numbered]
%\usepackage[default]{sourcesanspro}

\usepackage{tikz}
\usepackage{tikz-qtree-compat}
\usepackage{makecell}
\usepackage{pifont}
\usepackage{multirow}
\usepackage{graphicx}
\usepackage{xcolor}
\usepackage{hyperref}
\hypersetup{colorlinks,linkcolor=dgreen,urlcolor=dgreen,citecolor=dgreen}

\usepackage{gb4e}



 \newcommand{\foc}{\underline{\hspace*{12pt}} {} }		% DEFINES FOCUS BAR
%  \newfontfamily{\charis}{Charis SIL}
 \newcommand{\ctr}{\textsc{C}$_{\textrm{t}}$}
 \newcommand{\cpr}{\textsc{C}$_{\textrm{pro}}$}
 \newcommand{\gap}{\textsc{*Gap}}
%\newcommand{\shd}{\cellcolor[gray]{0.8}}
 \newcommand{\res}{\textsc{*Res}}
\newcommand{\soth}{\textsc{subject} vs. \textsc{other}}
\newcommand{\shd}{\cellcolor[gray]{0.8}}
\newcommand{\ra}{$\rightarrow$}
\newcommand{\gy}{\cellcolor[gray]{0.8}}
\newcommand{\wa}{ə}



\definecolor{bl}{HTML}{0C6FAB}
\definecolor{bgblue}{HTML}{AECDF0}
\definecolor{hlblue}{HTML}{0C6FAB}
\definecolor{bgbluet}{HTML}{455260}
\definecolor{tgr}{HTML}{597F5A}
\definecolor{gr}{HTML}{257427}
\definecolor{hl}{HTML}{cc00b9}
\definecolor{gree}{HTML}{95D496}
\definecolor{pp}{HTML}{b427ae}
\definecolor{rd}{HTML}{d22303}






\title{Numerus Morphologie und Merkmale: Teil I}
\author{Peter W. Smith}
\institute{p.smith@em.uni-frankfurt.de}
\date{25.10.2019, Aspekte der Flexionsmorphologie der Substantive}


\begin{document}

\maketitle

% \frame{\frametitle{Zusammenfassung}
% \begin{itemize}
% \item Letzte Woche haben wir:
%   \begin{itemize}
%   \item den Begriff ,,Wort'' besprochen.
%     \item den Unterschied zwischen Lexem und Wortform besprochen.
%   \item gelernt, dass `Morphem' nützlicher als `Wort' für Linguisten ist.
%     \item den Unterschied zwischen Präfixe, Suffixe, Infixe und Zircumfixe gesehen.
%     \end{itemize}
%     \item  Welche Morpheme gibt es in diesem Satz?
% \end{itemize}
%     \begin{exe}
% \ex
% Die Frau hat den Chefs einen Brief geschickt.
% \end{exe}


% }



% \section{Morphologische Merkmale}

\frame{\frametitle{Warum Merkmale?}
\begin{itemize}
\item Oft gibt es verschiedene Formen, die die gleiche Bedeutung tragen. Zum Beispiel Plural im Deutschen:
\end{itemize}

\begin{exe}
\ex
\begin{xlist}
\ex -en: die Studenten, die Themen.
\ex -e: die Friseure, die Hände
\ex -er: die Wörter
\ex -s: die Autos, die Hobbys
\ex \emph{keine}: die Löffel
\end{xlist}
\end{exe}
}

\frame{\frametitle{Warum Merkmale?}
\begin{itemize}
\item Obwohl die Wörter verschiedene Formen haben, finden wir die gleichen syntaktischen Muster.
\end{itemize}
\begin{exe}
  \ex
  \begin{xlist}
    \ex \textbf{Die} Studenten \textbf{sind} hier.
    \ex \textbf{Die} Friesuere \textbf{sind} hier.
    \ex \textbf{Die} Wörter \textbf{sind} hier.
    \ex \textbf{Die} Autos \textbf{sind} hier.
    \ex \textbf{Die} Löffel \textbf{sind} hier.
  \end{xlist}
\end{exe}

\begin{itemize}
\item Es gibt keinen syntaktischen Effekt der verschiedenen Realisierungen von Wörter.
\item In der Syntax sind Wörter abstrakt.
  \item Wir nennen diese abstrakten Formen ,,Merkmale''.
\end{itemize}
}

\section{Merkmale der Numerus}

\frame{\frametitle{Numerus}

\begin{itemize}
\item Der Numerus kodiert die Zahl der Referenten für einen Ausdruck.

\item Im Deutschen und Englischen (und in  den meisten europäischen Sprachen) gibt es nur Singular und Plural.

%\begin{itemize}
%\item Singular: eine Einheit.
%\begin{itemize}
%\item der Hund
%\item[{}] \includegraphics[width=5cm]{zeb.jpg}
%\end{itemize}
%\item Plural: mehr als eine Einheit.
%\begin{itemize}
%\item die Hunde
%\item[{}] \includegraphics[width=3cm]{zeb.jpg}
%\item[{}] \includegraphics[width=3cm]{zeb.jpg}
%\item[{}] \includegraphics[width=3cm]{zeb.jpg}
%%\item[{}] \includegraphics[width=3cm]{zeb2.jpg}
%%\includegraphics[width=2cm]{zeb2.jpg}
%\end{itemize}
%\end{itemize}

\item Das kann man durch Pronomen anschaulich machen:
\end{itemize}
\begin{table}\centering
\begin{tabular}[t]{l | l}
\hline
Singular	&	Plural\\
\hline
ich		&	wir\\
du		&	ihr\\
er/sie/es	&	sie\\
\hline
\end{tabular}
\caption{Singular vs. Plural in Pronimina}
\end{table}
}

\frame{\frametitle{Numerus}
\begin{table}\centering
\begin{tabular}[t]{l | l}
\hline
Singular	&	Plural\\
\hline
I		&	we\\
you		&	you\\
he/she/it	&	they\\
\hline
\end{tabular}
\caption{Pronomina im Englischen}
\end{table}

%\footnotetext{In some dialects in the southern states of the USA this pronoun can be realized as \emph{y'all}.}

}

\frame{\frametitle{Numerus: Das Dual}
\begin{itemize}
\item Es gibt auch andere Typen in den anderen Sprachen. Hier gibt es eine Kategorie für `dual':
\end{itemize}

\begin{exe}
\ex Amele (Papua Neuguinea)\\
\begin{xlist}
\ex
\gll Dana ho-i-a\\
man kommen-{\sc 3.singular-past}\\
\glt `Der Mann ist gekommen.'

\ex
\gll Dana ho-si-a\\
man kommen-{\sc 3.dual-past}\\
\glt `Die zwei Männer sind gekommen.'

\ex
\gll Dana ho-ig-a\\
man kommen-{\sc 3.plural-past}\\
\glt `Die Männer (\emph{x} \textgreater {} 2) sind gekommen.'
\end{xlist}
\end{exe}
}


\frame{
\frametitle{Forest Enets}
\begin{table}
  \centering
  \begin{tabular}[t]{l l l l}
    \hline
    &\textsc{singular}&\textsc{dual}&\textsc{plural}\\
    \hline
    1&modi/mod'	&modiniʔ	&medinaʔ\\
    2&ū&ūdiʔ&ūda\\
    3&bu&budiʔ&buduʔ\\
    \hline
  \end{tabular}
  \caption{Forest Enets Pronomina \citep{nsd}}
  \label{tab:forestenets}
\end{table}
}

%Es gibt auch Trial für drei Einheiten:
%
%\begin{exe}
%\ex Pronomen in Lihir (Papua New Guinea)\\
%\begin{tabular}[t]{l c c c c }
%\hline
%{}	&	Singular	&	Dual		&	Trial		&	Plural\\
%\hline
%2.	&	wa		&	gol		&	gotol		&	go\\
%3.	&	e		&	dul		&	dietol	& 	die\\
%\hline
%\end{tabular}
%\end{exe}

%Dual: exactly two items,
%Trial: exactly three items.
\frame{\frametitle{Numerus: Das Paucal}
\begin{itemize}
\item Einige Sprachen haben den Paukal, der ,,mehr als eins, aber nicht viel'' bedeutet.
\end{itemize}

\begin{exe}
\ex Bayso (äthiopien)
\begin{xlist}
\ex
\gll l\'{u}ban-titi foofe\\
löwe-{\sc singular} beobachten.{\sc 1.sg}\\
\glt `Ich beobachtete einen Löwe.'

\ex
\gll luban-jaa foofe\\
löwe-{\sc paucal} beobachten.{\sc 1.sg}\\
\glt `Ich beobachtete einige Löwen.'

\ex
\gll luban-jool foofe\\
löwe-{\sc plural} beobachten.{\sc 1.sg}\\
\glt `Ich beobachtete viele Löwen.'
\end{xlist}
\end{exe}
}



\frame{\frametitle{Die Merkmale des Numerus}
\begin{itemize}
\item Für eine Sprache wie Deutsch oder Englisch, die nur Singular und Plural hat, gibt es nur ein Merkmal: [$\pm$singular].
\end{itemize}

\begin{itemize}
\item {[$+$singular] = singular}
\item {[$-$singular] = plural }
\end{itemize}


\begin{figure}\centering
\begin{tikzpicture}[baseline]
\node at (3,1) {Number};
\node at (1,-1) {[$+$singular]};
\node at (5,-1) {[$-$singular]};
\draw (1,-0.8) -- (3,0.8) -- (5,-0.8);
\textcolor{red}{\draw<2> (1,-1) ellipse (1.5cm and 0.7cm);}
\textcolor{red}{\node<2> at (2,-2) {\textit{=singular}};}
\textcolor{red}{\draw<3> (5,-1) ellipse (1.5cm and 0.7cm);}
\textcolor{red}{\node<3> at (6,-2) {\textit{=plural}};}
%\node at (1,-2) {\sc singular};
%\node at (5,-2) {\sc plural};
%\draw (1,-1.3) -- (1,-1.8);
%\draw (5,-1.3) -- (5,-1.8);
\end{tikzpicture}
\caption{Singular vs. Plural}
\end{figure}
}

\frame{\frametitle{Die Merkmale des Numerus}
\begin{itemize}
\item Wenn eine Sprache den Dual hat gibt es ein weiteres Merkmal [Minimal].
\item {[Minimal]} bedeutet: ,,die kleinste Gruppe.''
\item Die Kombination von [$-$singular] und [Minimal] hat die Bedeutung ,,die kleinste Gruppe, die [$-$singular] ist''.
\item Diese Gruppe kann nur zwei Einheiten enthalten!
\end{itemize}
}

\frame{\frametitle{Die Merkmale des Numerus}
\begin{itemize}
\item In den \textsc{singular---dual---plural} Systemen hängt [Minimal] von [$-$singular] ab.
\end{itemize}

\begin{figure}\centering
\begin{tikzpicture}[baseline,scale=0.8]
\node (num) at (3,1) {Number};
\node (+sg) at (1,-1) {[$+$singular]};
\node (-sg) at (5,-1) {[$-$singular]};
%\draw (1,-0.8) -- (3,0.8) -- (5,-0.8);
\draw (+sg.north) to (num.south) to (-sg.north);
\node<1,4> (min) at (5,-2.5) {[Minimal]};
\draw<1,4> (min.north) to (-sg.south);
\textcolor{red}{\draw<2> (1,-1)  ellipse (1.5cm and 1cm);}
\textcolor{red}{\node<2> at (2.3,-2.3) {\textit{=singular}};}
\textcolor{red}{\draw<3> (5,-1) ellipse (1.5cm and 1cm);}
\textcolor{red}{\node<3> at (7.5,-1.5) {\textit{=plural}};}
\textcolor{red}{\draw<4> (5,-1.75) ellipse (1.5cm and 2cm);}
\textcolor{red}{\node<4> at (7.5,-1.75) {\textit{=dual}};}
%\node at (1,-2) {\sc singular};
%\draw (1,-1.3) -- (1,-1.8);
%\node at (3,-3) {[+Minimal]};
%\node at (7,-3) {[-Minimal]};
%\draw (3,-2.8) -- (5,-1.2};
%\draw (3,-3.2) -- (3,-4) node[below] {\sc dual};
%\draw (7,-3.2) -- (7,-4) node[below] {\sc plural};
%\node at (5,-2) {die Hunden};
%\draw (5,-1.3) -- (5,-1.8);
\end{tikzpicture}
\caption{Singular, Dual und Plural}
\end{figure}
%Wir können dies erklären im folgenden Art:
}

\frame{\frametitle{Die Merkmale des Numerus}
\begin{itemize}
\item Manchmal hat eine Sprache verschiedene Morpheme für [$-$singular] und [Minimal].
  \item z.B. Manam \citep{lichtenberk1983}:
\end{itemize}
\begin{columns}
  \begin{column}{0.5\textwidth}
\begin{exe}
  \ex
  \begin{xlist}
    \ex
     \'{a}ine \textbf{ŋ\'{a}ra}\\
    `die Frau'
    \ex
    \'{a}ine \textbf{ŋ\'{a}ra-di}\\
    `die Frauen'
    \ex
    \'{a}ine \textbf{ŋara-d\'{i}-aru} \\
    `die zwei Frauen'
\end{xlist}
\end{exe}
\end{column}
\begin{column}{.5\textwidth}
  \begin{itemize}
  \item \'{a}ine = frau
  \item ŋ\'{a}ra = die
  \item di = [$-$singular]
    \item aru = [Minimal]
  \end{itemize}
\end{column}
\end{columns}

}

\frame{\frametitle{Die Merkmale des Numerus}
\begin{table}\centering
\begin{tabular}[t]{l lllll}
\hline
	&	{\sc singular}	&	{\sc plural} 	& {\sc dual}	&	\\%{\sc paukal}\\
  \hline
  \sc 1excl	& iau	&	gim	& gi-ur	\\%	gim-tul		\\

\sc 1incl	& 	&	git	& git-ar\\%	&	git-tul	\\
2		& i\'au	&	gam	& ga-ur	\\%	gam-tul		\\
3 		& -i/on/\'ai	&	di	& di-ar	\\%	di-tul		\\
\hline
\end{tabular}
\caption{Pronomina des Sursurunga}
\end{table}

\begin{itemize}
\item g- = Basus
\item -i(t/m)/am = [$-$singular]
  \item -ur/ar = [Minimal]
\end{itemize}
}

\frame{\frametitle{Die Merkmale des Numerus}

\begin{itemize}
\item Mit dieser Theorie können wir die folgende Generalisierung erklären:
\end{itemize}


\begin{exe}
  \ex Es gibt keine Sprache, die einen Dual ohne Plural hat \citep{greenberg1963}.
\end{exe}

\begin{itemize}
\item<2-> Auch für den Paukal\ldots
\end{itemize}

  \begin{exe}
\ex<2-> Alle Sprachen, die Paukal haben, haben auch Plural \citep{corbett2000}.

\end{exe}
}




\frame{\frametitle{Die Merkmale des Numerus}
\begin{itemize}
\item Wenn eine Sprache den Paukal hat, gibt es ein Merkmal [Wenig].
\item {[Wenig]} bedeutet ,,eine kleine Gruppe.''
\item Die Kombination von [$-$singular] und [Wenig] hat die Bedeutung ,,eine Kleine Gruppe, die [$-$singular] ist''.
\item Wie [Minimal], {[Wenig]} hängt auch von [$-$singular] ab!

\end{itemize}
}


\frame{
\frametitle{Die Merkmale des Numerus}
\begin{figure}\centering
\begin{tikzpicture}[baseline,scale=0.8]
\node (num) at (3,1) {Number};
\node (+sg) at (1,-1) {[$+$singular]};
\node (-sg) at (5,-1) {[$-$singular]};
%\draw (1,-0.8) -- (3,0.8) -- (5,-0.8);
\draw (+sg.north) to (num.south) to (-sg.north);
\node<1,4> (min) at (5,-2.5) {[Wenig]};
\draw<1,4> (min.north) to (-sg.south);
\textcolor{red}{\draw<2> (1,-1)  ellipse (1.5cm and 1cm);}
\textcolor{red}{\node<2> at (2.3,-2.3) {\textit{= singular}};}
\textcolor{red}{\draw<3> (5,-1) ellipse (1.5cm and 1cm);}
\textcolor{red}{\node<3> at (7.5,-1.5) {\textit{= plural}};}
\textcolor{red}{\draw<4> (5,-1.75) ellipse (1.5cm and 2cm);}
\textcolor{red}{\node<4> at (7.5,-1.75) {\textit{= paukal}};}
%\node at (1,-2) {\sc singular};
%\draw (1,-1.3) -- (1,-1.8);
%\node at (3,-3) {[+Minimal]};
%\node at (7,-3) {[-Minimal]};
%\draw (3,-2.8) -- (5,-1.2};
%\draw (3,-3.2) -- (3,-4) node[below] {\sc dual};
%\draw (7,-3.2) -- (7,-4) node[below] {\sc plural};
%\node at (5,-2) {die Hunden};
%\draw (5,-1.3) -- (5,-1.8);
\end{tikzpicture}
\caption{Singular, Paucal und Plural}
\end{figure}
%\begin{figure}\centering
% \begin{tikzpicture}[baseline]
% \node at (3,1) {Number};
% \node at (1,-1) {[+singular]};
% \node at (5,-1) {[-singular]};
% \draw (1,-0.8) -- (3,0.8) -- (5,-0.8);
% \node at (1,-2) {\sc singular};
% \draw (1,-1.3) -- (1,-1.8);
% \node at (3,-3) {[+Wenig]};
% %\node at (7,-3) {[-Wenig]};
% \draw (3,-2.8) -- (5,-1.2);
% \draw (3,-3.2) -- (3,-4) node[below] {\sc paucal};
% %\draw (7,-3.2) -- (7,-4) node[below] {\sc plural};
% %\node at (5,-2) {die Hunden};
% %\draw (5,-1.3) -- (5,-1.8);
% \end{tikzpicture}
% \caption{Mit Paucal}
% \end{figure}
}

\frame{\frametitle{Die Merkmale des Numerus}
\begin{itemize}
\item Es ist möglich, dass eine Sprache {\sc paukal} und {\sc dual} hat:
\end{itemize}

\begin{figure}\centering
\begin{tikzpicture}[baseline,scale=0.8]
\node (num) at (3,1) {Number};
\node (+sg) at (1,-1) {[$+$singular]};
\node (-sg) at (5,-1) {[$-$singular]};
%\draw (1,-0.8) -- (3,0.8) -- (5,-0.8);
\draw (+sg.north) to (num.south) to (-sg.north);
\node (wen) at (6.5,-2.5) {[Wenig]};
\node (min) at (3.5,-2.5) {[Minimal]};
\draw (min.north) to (-sg.south);
\draw (wen.north) to (-sg.south);
% \textcolor{red}{\draw<2> (1,-1)  ellipse (1.5cm and 1cm);}
% \textcolor{red}{\node<2> at (2.3,-2.3) {\textit{=singular}};}
% \textcolor{red}{\draw<3> (5,-1) ellipse (1.5cm and 1cm);}
% \textcolor{red}{\node<3> at (7.5,-1.5) {\textit{=plural}};}
% \textcolor{red}{\draw<4> (5,-1.75) ellipse (1.5cm and 2cm);}
% \textcolor{red}{\node<4> at (7.5,-1.75) {\textit{=paucal}};}
%\node at (1,-2) {\sc singular};
%\draw (1,-1.3) -- (1,-1.8);
%\node at (3,-3) {[+Minimal]};
%\node at (7,-3) {[-Minimal]};
%\draw (3,-2.8) -- (5,-1.2};
%\draw (3,-3.2) -- (3,-4) node[below] {\sc dual};
%\draw (7,-3.2) -- (7,-4) node[below] {\sc plural};
%\node at (5,-2) {die Hunden};
%\draw (5,-1.3) -- (5,-1.8);
\end{tikzpicture}
\caption{Singular, Paukal und Plural}
\end{figure}

% \begin{figure}\centering
% \begin{tikzpicture}[baseline]
% \node at (3,1) {Number};
% \node at (1,-1) {[+singular]};
% \node at (5,-1) {[-singular]};
% \draw (1,-0.8) -- (3,0.8) -- (5,-0.8);
% \node at (1,-2) {\sc singular};
% \draw (1,-1.3) -- (1,-1.8);
% \node at (2.5,-3) {[+Wenig]};
% %\node at (8,-3) {[-Wenig]/[-Minimal]};
% \draw (3,-2.8) -- (5,-1.2);
% \draw (3,-3.2) -- (2.5,-4) node[below] {\sc paucal};
% %\draw (8,-3.2) -- (9,-4) node[below] {\sc plural};
% \draw (5,-1.2) -- (5,-2.7) node[below] {[+Minimal]};
% \draw (5,-3.2) -- (5,-4) node[below] {\sc dual};
% %\node at (5,-2) {die Hunden};
% %\draw (5,-1.3) -- (5,-1.8);
% \end{tikzpicture}
% \caption{Beides}
% \end{figure}
}

\frame{
\frametitle{Boumaa Fijian}
\begin{table}
  \centering
  \begin{tabular}{l l l l l}
    \hline
    Person&\textsc{singular}&\textsc{dual}&\textsc{paukal}&\textsc{plural}\\
    \hline
    1excl&au&'eirau&'eitou&'eimami\\
    1incl&au&(e)tauru&tou&(e)ta\\
    2&o&(o)mudrau&(o)mudou&(o)munuu\\
    3&e&(e)rau&(e)ratou&(e)ra\\
    \hline
  \end{tabular}
  \caption{Boumaa Fijian \citep{Dixon88}}
  \label{tab:fijian}
\end{table}
}

\section{Weitere Kategorien}

\begin{frame}
    \frametitle{Trial}
        \begin{itemize}
            \item Manche (wenig) Sprachen haben den Trial, mit der Bedeutung "drei Einheiten."
        \end{itemize}
        \begin{table}
          \centering
          \begin{tabular}[t]{l l l l l l}
            &\textsc{singular}&\textsc{dual}&\textsc{trial}&\textsc{paucal}& \textsc{plural}\\
            \hline
            \textsc{1excl}&yo&gel&getol&gehet&ge\\
            \textsc{1incl}&---&kito&kitol&kitahet&giet\\
            2&wa&gol&gotol&gohet&go\\
            3&e&dul&dietol&diehet&die\\
            \hline
          \end{tabular}
        \caption{Lihir}
        \end{table}
        \begin{itemize}
          \item Bedeutung von plural: ``größer als Puakal.''
        \end{itemize}
\end{frame}

\begin{frame}
  \frametitle{Trial}
  \begin{itemize}
    \item Wie erklären wir Trial?
    \item<2-> Ein weitere Merkmal?
    \item<3-> Reihenfolge der Komposition der Merkmale?
  \end{itemize}
\end{frame}

\begin{frame}
  \frametitle{Folge der Komposition}
  \begin{itemize}
    \item Numerusmerkmale haben eine klare Bedeutung.
    \item {[±Singular] = singular oder nicht singular}
    \item {[±Minimal] = kleinste Gruppe oder nicht}
    \item<2-> Wahrscheinlich ist die Bedeutung von Substantive alle mögliche Gruppierungen von der Substantiv.
    \item<3-> Für viele Fälle, die Folge der Kompositionhat kein Effekt.
  \end{itemize}
\end{frame}

\begin{frame}
  \frametitle{Folge der Komposition}
  \begin{itemize}
    \item Aber für den Trial gibt es tatsächlich ein Effekt.
    \item<2-> Root + [−Singular] = nicht singular Gruppen
    \item<3-> Root + [−Singular] + [−Minimal] = alle nicht singular Gruppen, die mehr als \textbf{zwei} Einheiten haben.
    \item<4-> Root + [−Singular] + [−Minimal] + [+Minimal] = alle Gruppen mit \textbf{drei} Einheiten.
  \end{itemize}
\end{frame}

\begin{frame}
  \frametitle{Folge der Komposition}
  \begin{itemize}
    \item Andere Frage: benutzen jede Sprache alle Merkmale?
    \item<2-> \textbf{Nein}: im Englischen gibt es kein Dual, also kein [±Minimal].
    \item<3-> Und [±Singular]?
  \end{itemize}
\end{frame}

\begin{frame}
  \frametitle{Nur [±Minimal]}
    \begin{itemize}
      \item Im Ilocano scheint es, dass es nur Dual im \textsc{1.inklusiv} gibt.
    \end{itemize}
    \begin{table}
      \caption{Ilocano: Traditionelle Beschreibung}
      \label{}
      \begin{tabular}{lccc}
        \hline
        &\textsc{singular}&\textsc{dual}&\textsc{plural}\\
        \hline
        \textsc{1excl}&ko&---&mi\\
        \textsc{1incl}&---&ta&tayo\\
        \textsc{2}&mo&---&yo\\
        \textsc{3}&na&---&da\\
        \hline
      \end{tabular}
    \end{table}
    \begin{itemize}
      \item<2-> Aber, warum?
      \item<3-> Eine bessere Analyse würde ``warum'' erklären.
    \end{itemize}
\end{frame}

\begin{frame}
  \frametitle{Nur [±Minimal]}
  \begin{itemize}
    \item Analyse: Ilocano benutzt \textbf{nur} [±Minimal].
    \item Wir nennen dieses System `Minimal--Augmented'.
  \end{itemize}
  \begin{table}
    \caption{Ilocano: Minimal--augmented}
    \label{}
      \begin{tabular}{lcc}
        \hline
        &[+Minimal]&[−Minimal]\\
        \hline
        1&ko&mi\\
        1+2&ta&tayo\\
        2&mo&yo\\
        3&na&da\\
        \hline
      \end{tabular}
  \end{table}
\end{frame}


\begin{frame}[allowframebreaks]
  \frametitle{References}
\printbibliography
\end{frame}
\end{document}
